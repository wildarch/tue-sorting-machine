\documentclass[a4paper,oneside,11pt]{article}
\usepackage{a4wide,graphicx,fancyhdr,amsmath,amssymb}

%-------- Macros and Definitions --------%

\setlength\headheight{20pt}
\addtolength\topmargin{-10pt}
\addtolength\footskip{20pt}

\newcommand{\subject}{2IO70 embedded systems}

\newcommand{\N}{\mathbb{N}}
\newcommand{\ch}{\mathcal{CH}}

\newcommand{\naam}{Reinhard Heinrich Bertram Vinzenz Freiherr von Pelden genannt Cloudt zu Lauersfort und Impel}

\fancypagestyle{plain}{%
\fancyhf{}
\fancyhead[LO,RE]{\sffamily\bfseries\large technische universiteit eindhoven}
\fancyhead[RO,LE]{\sffamily\bfseries\large \subject}
\fancyfoot[LO,RE]{\sffamily\bfseries\large department of mathematics and computer science}
\fancyfoot[RO,LE]{\sffamily\bfseries\thepage}
\renewcommand{\headrulewidth}{0pt}
\renewcommand{\footrulewidth}{0pt}
}

\pagestyle{fancy}
\fancyhf{}
\fancyhead[RO,LE]{\sffamily\bfseries\large technische universiteit eindhoven}
\fancyhead[LO,RE]{\sffamily\bfseries\large \subject}
\fancyfoot[LO,RE]{\sffamily\bfseries\large department of mathematics and computer science}
\fancyfoot[RO,LE]{\sffamily\bfseries\thepage}
\renewcommand{\headrulewidth}{1pt}
\renewcommand{\footrulewidth}{0pt}

\usepackage[margin=1in]{geometry}
\usepackage{float}
\usepackage{subcaption}
\usepackage{graphicx}

%-------- Title --------%

\title{\vspace{-\baselineskip}\sffamily\bfseries Machine Design}
\author{
	\makebox[.25\linewidth]{Sergio van Amerongen}\\0952200 \and
	\makebox[.25\linewidth]{Stefan Cloudt}\\0940775 \and
	\makebox[.25\linewidth]{Daan de Graaf}\\0956112 \and
	\makebox[.25\linewidth]{Robert van Lente}\\0953343 \and
	\makebox[.25\linewidth]{Tom Peters}\\0948730 \and
	\makebox[.25\linewidth]{Berrie Trippe}\\0948147 
	\and \makebox[.75\linewidth]{\textbf{Responsible:}} \and
	Robert van Lente\\ \tt{r.f.v.lente@student.tue.nl}
}
\date{\today}

%-------- Document --------%

\begin{document}
\maketitle

\section{Introduction}
We started our project by determining the features that we wanted to implement in our sorting machine, like user interaction and exception handling. These design decisions form the foundation of our mechanical machine design. The purpose of this document is to record the design decisions we made in the first stage of the project, from a mechanical point of view as well as
a user point of view.

\section{Design requirements}
In order to create a functioning sorting machine that can sort black and white discs, at least the following components are required.
\begin{itemize}
\item A container for storing unsorted discs
\item Two containers for storing the sorted discs
\item A sensor that can distinguish the different colour discs.
\item A transporter that is able to move the discs from the container to the sensor.
\item A transporter that is able to move the discs from the sensor to the appropriate container.
\end{itemize}

\section{Final Design}
The machine consists of a vertical tube in which the discs can be stored. A pin can be removed from the bottom of the tube, allowing the discs to fall on a platform below, one by one. A colour sensor is attached underneath this platform, such that it can detect the colour of the disc.
Next to this platform, a step motor is attached that rotates a wheel with three arms. These arms can push the disc of the platform, either on the left or the right side.
Below the platform, a seesaw is attached with both sides of the same length, such that the discs that fall on the platform can fall on these arms. A gyroscope is attached to the axis of this seesaw, such that rotations of the seesaw can be detected.
The discs then slide off the seesaw, allowing it to stabilize, and fall into one of two mountable trays.

\newpage

\section{Design Decisions}
The machine went through several iterations before we settled on the final design. The decisions that we made during the design process are listed below.

\subsection{Position of the colour sensor}
We had the option to attach the colour sensor in various ways to the machine. We chose to put it underneath the platform with the sensor pointing up, in such way that the arm of the motor could easily swipe the discs away.  Placing the sensor below the discs makes the sensor more reliable, since the surface of the discs on the bottom is larger than the surface on the side of the discs. It would also have been significantly more difficult to mount the colour sensor in a sideways or upright position because of the structural limitations of the lego construction set.

\subsection{Container}
The decision to use a tube-like structure was made early on, as it is the most compact way to store the discs. The vertical storage also has the added benefit of removing the need to create an additional transporter to move the discs from the container to the sensor, since a new disc will simply fall in place when the previous disc is moved away.

The height between the tube and the platform is designed to be slightly higher than the height of one disc. This ensures that only one disc will be moved from underneath the tube at a time.

\subsection{Rotating wheel}
The sorting wheel was initially equipped with four arms. We later switched to using three arms since it allowed for simple positioning of the touch sensor for calibration, as well as being easier to construct.
 
\subsection{Seesaw}
Initially, the plan was to use touch sensors to detect if a disc arrived in the correct tray. However, we quickly found that the mass of the discs was too small to trigger the touch sensors. Other available sensor proved irreliable, hence we eventually settled on using the seesaw.

\subsection{Trays}
We chose to use coffee cups for the trays in the machine, since these where readily and freely available. Other options would be buying plastic trays or building trays out of LEGO, which was not possible because there was not a sufficient number of parts in the construction set.

\begin{figure}[H]
\begin{subfigure}{0.5\textwidth}
	\centering
	\includegraphics[width=80mm]{front.png}
	\caption{\label{firstfront}Front view}	
\end{subfigure}	
\begin{subfigure}{0.5\textwidth}
\centering
	\includegraphics[width=80mm]{top.png}
	\caption{\label{top}Top view.}
\end{subfigure}	
\caption{The initial design schematics of the machine.}
\end{figure}

\begin{figure}[H]
	\centering
	\includegraphics[width=90mm]{frontnew.png}
	\caption{\label{front}Front view of the final machine.}
\end{figure}

\section{System Level Requirements}
\subsection{Use Cases}
The machine can be described as a finite automaton. This finite automaton is shown in figure \ref{machinedesign}. The machine has three buttons, which the user can utilize to control the machine. The
START/PAUSE button is used to start and pause the machine. The ABORT button is used to
stop the machine in case of emergency. The RESET button is used to reset the machine. The
user is required to add black or white discs to the container before starting the machine.
The machine, when finished without fatal error, produces two trays, one filled with white
and one filled with black discs. The machine will also display information about its current
state and progress on the display during the sorting process.

The sorting process is described by the following automaton. The initial state is the mode selection state. In this state, the mode in which the machine operates can be selected. These modes are explained further in the document. After the mode has been selected, the machine goes to the initial state of the sorting process. This initial state is the resting state. The user is required to prepare the machine as described in the User Constraints segment further in the document. None of the buttons should have any effect, except for the START/PAUSE button, which starts the sorting process and the ABORT button, which triggers an abort. When this button is pressed, the machine should proceed to the operating state.

In this state, the machine sorts the discs. Internally, it has a state in which it checks the
color of the disc, after which it moves to the correct sorting state in which it transports the disc
to the correct tray. If the RESET button is pressed in this state, the machine should return to
the resting state after finishing its current cycle.

While in the operating state, if the START/PAUSE is pressed, the process should halt after it has finished its current cycle 
and the machine should go into a paused state. When the START/PAUSE button is pressed
again, the machine should return to the operating state at the point where it was when the
START/PAUSE button was first pressed. If the RESET button is pressed in this pausing state,
the machine should return to the resting state.

While in the operating state, if the ABORT button is pressed or a fatal exception is encoun-
tered, the machine should immediately go into an exception state. In this state, the machine
must come to a full stop. The machine should only continue when the RESET button is pressed,
after which it returns to the mode selection state.

When the machine finds that there are no more discs left in the storage, it should go into a
finished state. Then, if the RESET button is pressed, the machine should return to the mode selection
State.

\begin{figure}[H]
	\centering
	\includegraphics[width=125mm]{machinedesign.png}
	\caption{\label{machinedesign}The finite state automaton of the machine.}
\end{figure}

\subsection{User Constraints}
\subsubsection{Machine Preparation}
Before starting the operation of the machine with the START/PAUSE button in the resting state, the user is expected to fill the disc storage with black and
white colored discs that need to be sorted. Only discs are allowed in the disc storage and no
other objects should be placed in it. The user must also ensure that no discs are present in
other parts of the machine and that the disc trays are mounted to their mounting points prior
to starting execution. The disc trays should be empty. At most 12 discs may be placed into the
container. The user then has to proceed by starting the program. At this point, the machine will calibrate the arm to the standard position. After the calibration, the tube can be filled with
discs by pulling out the plug.

After the machine has finished execution, the user will have access to two trays, one containing exclusively white discs, the other only contains black discs, in the case everything goes as expected. When the machine indicates
that it has finished its sorting procedure, the user should remove the trays from their mounts
so he can dispose the discs somewhere else.

\subsubsection{Exceptions}
Our machine should be capable of detecting the following errors:
\begin{figure}[H]
\centering
\begin{tabular}{|l|l|}
\hline
Error type & Fatal (= go to exception state)\\\hline
Disc does not reach the tray & Yes\\\hline
Time to tray is higher than average & No\\\hline
Disc arrives in the wrong tray & No\\\hline
Wrong input (i.e. different colour disc) & No\\\hline
Motor jams & Yes \\\hline
Gyroscope does not stabilise & Yes \\\hline
Battery low & Yes \\\hline
Abort by user & Yes \\\hline
Software exception & Yes \\\hline
\end{tabular}
\end{figure}
When a disc does not reach the tray, the machine should stop, as there is may be an obstruction in the path the disc takes from the sorting arm to the lever.

When a disc arrives in the wrong tray, the machine should stop, as something happened that caused the disc to move to the wrong tray, even though the sorting arm tried to move it to the correct tray. This means there must be a mechanical failure somewhere. However, because of the limitations the gyroscope and the API have given us, we can not trust this check entirely. Therefore, we have decided to make this a non-fatal error instead, so the machine does not halt when it should not. The machine should also detect when the time it takes for a disc to reach the tray takes longer than usual, as this could indicate that some obstruction may be present.

When a disc is detected with a color other than white
or black, the user has input wrong discs. This is not a fatal error because
we can specify an action that will be taken when this exception occurs. We may for example
want the machine to simply put the unknown disc in one of the trays and notify the user of this, or dispose all unknown discs into a separate tray.

When the motor is jammed or the gyroscope does not stabilize, the machine should stop immediately, so that the user can remove whatever is blocking the motor or the gyroscope.

The machine will give a warning when the battery has less than 5\% of charge left. This
is done to prevent the machine running out of power in between operations and ending in an
invalid state. The user should act appropriately when this happens.

The machine should obviously stop when the user presses the abort button.

When the software in the machine encounters an exception, the machine should stop running, so that nothing goes wrong and the error can be corrected.

When an error occurs in the machine that is indicated with fatal, the user must remove all
discs from all parts of the machine. To help the user with this, the machine has a build-in function that spits out all the discs to the right when pressing the ABORT  button in the mode selection state. The user should first make sure this can be done safely and without harming the machine. The user should then press the RESET button in order
to put the machine in the mode selection state again. The user may then proceed to prepare the machine for
execution again, as described above, if desired.

\paragraph{Safety properties}
The machine should have the following safety properties. These are our
guarantees about the working and stopping conditions of the machine, in case something unex-
pected happens.

\begin{itemize}
	\item The ABORT button must stop all moving parts within 10 ms. This to make sure that the user is able to quickly stop the machine in case something can not be handled by the machine.
	\item A fatal exception must stop all moving parts within 10 ms. This ensures that the machine
does not damage or even destroy itself when it is blocked in any way.
	\item When the machine starts with discs loaded into the container, then the discs will be sorted
when it arrives in the finished state. The sorting process takes at most 5 minutes in the
default safe mode.
\end{itemize}

\section{Machine Interface}
\subsection{Lejos API}
The Lejos API provides access to buttons on the brick, sensors and actuators.
Buttons can be queried directly, but references to sensors and actuators must be instantiated
with a ’Port’ parameter indicating how they are connected to the brick. The following ports
are used in our design:

\begin{figure}[H]
\begin{tabular}{|l|l|}
\hline
\textbf{Peripheral} & \textbf{Port} \\
\hline
Colour sensor & Port 1 \\
Gyroscope & Port 2 \\
Touch sensor & Port 3 \\
Motor & Port A \\
\hline
\end{tabular}
\end{figure}

Motors expose methods that allow the developer to have to motor rotate an arbitrary number of degrees. Therefore, it is not necessary to implement pulse width modulation. One can call a method to fetch a sample from a sensor in the form of an array of floats. It is up to the developer to parse these in a meaningful way.

\subsection{Peripherals}

\subsubsection{Motor rotating arm}
This motor is connected to port A and rotates the arm which moves the discs. The arm has three legs with an angle of 120 degrees. The motor has basically three states: a resting state and two working states, one for every possible direction. The neutral position of the three legged arm is the position where the two arms closest to the place where the disc will land have an approximately identical distance to the disc.

When the motor is in the resting state the three legged wheel has this neutral position. When the motor is in a working state the arms rotate left or right until the next neutral position of the wheel (120 degrees further), where it will transition to the resting state of the motor. During the working state the motor should not be jammed or obstructed in any way, or the motor will be jammed and trigger an exception.

\subsubsection{Touch sensor calibrator}
The touch sensor, connected to port 3, is attached in such a way behind the sorting platform that the arms of the sorting wheel are able to fully press the sensor when it rotates. This way, it is possible to detect the position of the sorting wheel when it is rotating, which can be used to calibrate the sorting wheel through software.

\subsubsection{Gyroscope with seesaw}
To the seesaw a gyroscope is attached, which is wired to port 2. The seesaw has three states, a neutral state, a left-side down state and a right-side down state. The neutral state is the state in which the gyroscope when read returns a value indicating that it is in balance. The seesaw enters a left-side down state when a disc falls on the left side. The gyroscope will read a value indicating that it was unbalanced and that the left side went down.
This is analogous for the right side down state. The gyroscope is unreliable at best: Its value tends to drift at an arbitrary rate, making the gyroscope unreliable if not unusable in some cases. While the seesaw is in motion no further discs should be sorted, as it will not be possible to accurately measure the angle rate change. Extensive testing suggests, although it is not documented anywhere, that the gyro sensor is much more reliable if one at first calibrates the sensor by quickly switching from rate change mode to angle\footnote{The gyro sensor has both an angle and a rate change mode, which return the measured angle and the rate of change of this angle, respectively. The angle mode is so unreliable that it is considered unusable for our purposes} and back. If the sensor is then kept completely stable, a lower drift rate may be acquired.

\subsubsection{Colour sensor}
The colour sensor, connected to port 1, gives a value indicating the grayscale of the surface above it. The colour sensor returns a float indicating the grayscale (e.g. 0 uptil 1) of the disc currently on the platform, or the grayscale of the environment, which means that there is no disc above it. After extensive testing, we found that the following values yield the correct result from the machine:

\begin{figure}[H]
\begin{tabular}{|l|l|}
\hline
\textbf{Colour disc} & \textbf{Value} \\
\hline
None & 0.12 \\
Black & 0.15 \\
White & 0.86 \\
\hline
\end{tabular}
\end{figure}

However, the user should be able to calibrate the machine in the mode selection state, before running the sorting process, by placing a black disc and a white disc on the sensor and running a calibration sequence. The grayscale values should be changed according to the values found. If no calibration is done, the values of last calibration will be used.

\newpage

\subsubsection{The brick}
The brick controls the sensors and actuators. The state of the buttons is either
up or down, which may be queried using the Lejos API. The buttons used are shown in figure \ref{brickbuttons}.
\begin{figure}[H]
	\centering
	\includegraphics[width=60mm]{BrickButtons.png}
	\caption{\label{brickbuttons}The buttons on the brick.}
\end{figure}

\section{Conclusion}
The design decisions recorded in this chapter will be built upon in the next phase of the
project, where we are going to specify the software specifications. The software specifications
are heavily dependent on the machine interface described in this chapter, since the machine
interface describe our assumptions about the machine API and the mechanical implementation.
The software specifications will describe how we are going to use this interface and how the
machine will behave.
\end{document}
